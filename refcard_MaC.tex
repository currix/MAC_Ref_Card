\documentclass[10pt]{article}
\usepackage{fixltx2e}
\usepackage[orthodox,l2tabu,abort]{nag}
\usepackage[utf8]{inputenc}
% Page layout
\usepackage[landscape,margin=0.5in]{geometry}
\usepackage{multicol}

% Title area
\usepackage{titling} % Allows for use of date, author, etc. after \maketitle
% Ref: http://tex.stackexchange.com/questions/3988/titlesec-versus-titling-mangling-thetitle
\let\oldtitle\title
\renewcommand{\title}[1]{\oldtitle{#1}\newcommand{\mythetitle}{#1}}
\renewcommand{\maketitle}{%
{\begin{center}\Large \mythetitle\end{center}}
}

% Document divisions
\usepackage{titlesec}
\setcounter{secnumdepth}{0}
\titlespacing{\section}{0pt}{0pt}{0pt}
\titlespacing{\subsection}{0pt}{0pt}{0pt}
\usepackage{nopageno} % To keep \section from resetting page style

\setlength{\parindent}{0pt} % disabling indentation by default

% Lists
\usepackage{enumitem} % for consistent formatting of lists
\newlist{ttdesc}{description}{1}
\setlist[ttdesc]{font=\ttfamily,noitemsep}
\usepackage{calc} % for \widthof

% Code
\usepackage{listings}
\lstset{language=[LaTeX]TeX,%
  basicstyle=\itshape,%
  keywordstyle=\normalfont\ttfamily,%
  morekeywords={part,chapter,subsection,subsubsection,paragraph,subparagraph}%
  }

\usepackage{amsmath} % Required for some math elements 
\usepackage[utf8]{inputenc} % Required for special characters processing
\usepackage{siunitx} % Provides the \SI{}{} and \si{} command for typesetting SI units
\newcommand{\gc}{\degreeCelsius}

\usepackage{lipsum}

\title{Meteorology \& Climatology RefCard}
\author{Curro Pérez Bernal}
\date{2015-2016}

\begin{document}
\begin{multicols}{3}
\maketitle

\section{Physics and Thermodynamics}

\subsection{Basic Thermodynamics}
\begin{ttdesc}[labelwidth=\widthof{\texttt{report}}]
\item[Ideal gas Eq.] $p = \rho r T$~,~$r = \frac{R}{M}$
\item[Heat Capacity] $Q = m c \Delta T$
\item[First Princ.] $\Delta U = Q - W$
\item[Idem., diff.] $c_pdT = \frac{dp}{\rho}$, $c_vdT = -pdV$
\item[Latent Heat] $Q = m L$
\item[L Dependence with $T$] $L=L_0 + (c_{pw}-c) T (\si{\gc})$
\item[Poisson Eqs.] $\gamma = \frac{c_p}{c_v}\simeq 1.4~,~\kappa = \frac{r}{c_p}\simeq 0.286$, 
\begin{displaymath}
  p V^\gamma = \text{constant}~,~T p^{-\kappa} = \text{constant}
\end{displaymath}
\end{ttdesc}
\subsubsection{Constant values and units}
%
\begin{ttdesc}[labelwidth=\widthof{\ttfamily{letterpaper/a4paper}}]
\item[Universal Gas Constant] $R = \SI{8.314472}{J.mol^{-1}.K^{-1}}$
\item[Dry Air Gas Constant] $r_d = \SI{287}{J.kg^{-1}.K^{-1}}$
\item[Water Gas Constant] $r_w = \SI{461}{J.kg^{-1}.K^{-1}}$
\item[Dry air heat capacity] $c_{pd} = \SI{1006}{J.mol^{-1}.K^{-1}}$
\item[Water vapor heat capac.] $c_{pw} = \SI{1846}{J.mol^{-1}.K^{-1}}$
\item[Water latent heat of vap.] $L_{v} = \SI{2.257e6}{J.kg^{-1}}$
\item[Water latent heat of fuss.] $L_{f} = \SI{3.34e5}{J.kg^{-1}}$
\end{ttdesc}

\subsection{Radiation Heat Transfer}
\begin{ttdesc}[labelwidth=\widthof{\ttfamily{letterpaper/a4paper}}]
\item[Wien's Law] $\lambda_M T = \SI{2.898e-3}{m.K}$
\item[Stefan-Boltzmann Law] $M_e = \varepsilon \sigma T^4$
\end{ttdesc}


\subsubsection{Constant values and units}
\begin{ttdesc}[labelwidth=\widthof{\ttfamily{letterpaper/a4paper}}]
\item[Boltzmann Constant] $\sigma = \SI{5.6704e-8}{W.m^{-2}.K^{-4}}$ 
\end{ttdesc}

\section{Water Vapor and Humidity}
\begin{ttdesc}[labelwidth=\widthof{\ttfamily{letterpaper/a4paper}}]
\item[Moist air $r$ constant] $\bar r = q r_w + (1-q) r_d$.
\item[Relative Humidity] $h = 100\frac{e}{E}\simeq 100 \frac{a}{A} \simeq 100 \frac{m}{M}$
\item[Mixing Ratio] $\frac e P = \frac m {\epsilon + m}$
\item[Specific Humidity] $q=\frac{a}{\rho}$ and $q=\frac m {m+1}$
\item[Absolute Humidity] $e = a r_w T$
\item[Useful Relations]  $\frac{de}{e}=\frac{dm}{m}+\frac{dP}{P}$ and
 $\frac{dh}{h}=\frac{dP}{P}-\frac{dE}{E}$.
\item[Clausius-Clapeyron Equation]
\begin{align*}
\frac{dE}{dT} &= \frac {L E}{r_w T^2} ~~\text{(differential form)} \\
\ln\frac E {E_0} &= \frac L {r_w} \left(\frac 1 {T_0} - \frac 1 T \right)\\
\ln\frac h {h_0} &= \frac L {r_w} \left(\frac 1 T - \frac 1 {T_0}\right) ~~\text{(only isobaric)}
\end{align*}
\item[Magnus formula] 
\begin{displaymath}
E(T) =  A \times exp{\frac{B\;  T[\si{\gc}]}{C + T[\si{\gc}]}} \ [\si{hPa}] \ .
\end{displaymath}
Water: $A = 6.1094, B = 17.625, C = 243.04$\\
Ice: $A = 6.1121, B = 22.587, C = 273.86$
\item[Virtual Temperature] 
\begin{displaymath}
\overline{r} T= r T_v \Rightarrow T_v=T \left( 1 + \frac 3 5 q \right)~.
\end{displaymath}
\item[Equivalent Temperature] 
\begin{displaymath}
T_e=T+\frac{m L}{c_p} \simeq T + 2 a \left( \si{g.m^{-3}} \right)~.
\end{displaymath}
\item[Wet-bulb Temperature]
\begin{align*} 
  (c_{pd}+m c_{pw}) (T-T_w) =& L [M(T_w)-m] \nonumber \\
T_e \simeq T_w + \frac{M(T_w) L}{c_{pd}} \simeq& T_h + 2 A(T_h)
\end{align*}
\end{ttdesc}
%
\subsubsection{Constant values and units}
%
\begin{ttdesc}[labelwidth=\widthof{\ttfamily{letterpaper/a4paper}}]
\item[Molecular mass ratio] $\epsilon = \frac{M_w}{M_d}=\frac {r_{d}}{r_{w}}=0.622$
\item[Water Gas Constant] $r_w = \SI{461}{J.kg^{-1}.K^{-1}}$
\end{ttdesc}

%\vspace{-\baselineskip}
%
%\begin{multicols*}{2}
%\lstinline|\part{title}| \\
%\lstinline|\chapter{title}| \\
%\lstinline|\section{title}| \\
%\lstinline|\subsection{title}| \\
%\lstinline|\subsubsection{title}| \\
%\lstinline|\paragraph{title}| \\
%\lstinline|\subparagraph{title}|
%\end{multicols*}

\section{Atmospheric Processes}

\begin{ttdesc}[labelwidth=\widthof{\ttfamily{letterpaper/}}]
\item[Potential Temperature] $\theta = \left(\frac {1000} P
  \right)^{r/c_p} T$
\item[Adiab.\ Elevation(linear)]  $T(z)=T_0 \left( 1 - \frac {\Gamma z} {T'_0} \right)$
\item[Adiab.\ Elevation(exact)] $T(z)=T_0 \left( 1-\alpha \frac z {T'_0} \right)^{\Gamma/\alpha}$
\item[Tropospheric Lapse Rate]  $T'(z)=T'_0 - \alpha z$
\item[Equilibrium Height] $z_e$ such that $T'(z_e) = T(z_e)$.
\item[Stability Index] $\eta=g \frac {\Gamma-\alpha}{T'}$
\item[$dh/dT$ in an adiabatic ascent]
\begin{displaymath}
\frac{dh}{dT}= \frac h T \left( \frac {\overline{c}_p}{\overline{r_d}} -\frac L {r_w T} \right) \ .
\end{displaymath}
\item[Exact $h(T)$ in an adiabatic ascent]
\begin{displaymath}
\ln\frac h {h_0} \ = \frac{{\overline c}_p}{\overline{r}} \ln\frac T {T_0} 
+ \frac {\epsilon L} r_d \left( \frac 1 T - \frac 1 {T_0} \right) \ .
\end{displaymath}
\item[Pseudoadiab.\ Ascent Lapse Rate]$-L dM \simeq c_p dT - V dP \nonumber$
\begin{displaymath}
  \Gamma_{pseud} = \Gamma \frac{P+\epsilon \frac{L E}{R T}}{P+\epsilon  \frac{L}{c_p}\frac{dE}{dT}}
\end{displaymath}


\item[Approximate h(T) in an adiabatic ascent]
\begin{displaymath}
\frac h {h_0} \ = \left( \frac T {T_0} 
\right)^{\displaystyle \frac{\overline{c}_p}{\overline{r}}-\frac{\epsilon L}{r_d T_0}} \ .
\end{displaymath}

\item[Ferrel Formula] $z_s=122 (T_0-\tau_0) ~~ (\si{m}) $
\item[V\"ais\"al\"a Formula] $z_s=188 \left(T(\si{\gc}) + 105 \right) \frac{\log_{10}\frac {100}{h_0}}{\log_{10}\frac {100}{h_0}+5.1}$

\end{ttdesc}
\subsubsection{Constant values and units}
%
\begin{ttdesc}[labelwidth=\widthof{\ttfamily{letterpaper/a4paper}}]
\item[Adiabatic Lapse Rate] $\Gamma = \frac{g}{c_{pd}}=\SI{9.8}{K.km^{-1}}$
\end{ttdesc}


\subsection{Polytropic Processes}
\begin{ttdesc}[labelwidth=\widthof{\ttfamily{letterpaper/a4paper}}]
\item[Polyt.\ Heat Capacity]  $\left\{ \begin{array}{c} c_p \to c_p-c \\
c_v \to c_v-c \end{array} \right.$
\item[Polyt.\ Heat Capacity] $\delta q = c dT$
\item[First Principle Polyt.] $({\bar c}_p - c) dT + \frac{T}{T'} g dz = 0$
\end{ttdesc}
%
\subsection{Winds}
\begin{ttdesc}[labelwidth=\widthof{\ttfamily{letterpaper}}]
\item[Gradient Pressure Accel.]  $\vec{a}_{p}=-\frac 1 \rho \vec{\nabla}P  \rightarrow -\frac 1 \rho \frac{dP}{dx}$
\item[Coriolis Accel.] $a_c=2 v \Omega \sin\phi$ where $\phi = $ 
  latitude and $\Omega  =2\pi/T_{rot}$ angular velocity.
\item[Geostrophic Vel.] $v_g=\frac{1}{\rho f}\frac{dP}{dx}$ where
$f=2 \Omega \sin\phi$.
\end{ttdesc}

%\begin{tabular}{lll}
%Command & Declaration & Effect \\
%\lstinline|\textrm{text}| & \lstinline|{\rmfamily text}| & \textrm{Roman family} \\
%\lstinline|\textsf{text}| & \lstinline|{\sffamily text}| & \textsf{Sans serif family} \\
%\end{tabular}

%\section{Math mode}

% \subsection{Math mode symbols}
% \begin{multicols*}{4}
% \( \leq \) \verb|\leq| \\
% \( \times \) \verb|\times| \\
% \( ^{\circ} \) \verb|^{\circ}| \\
% \( \infty \) \verb|\infty| \\
% \( \supset \) \verb|\supset| \\
% \( \subset \) \verb|\subset| \\
% \( \cup \) \verb|\cup| \\
% \( \dot a \) \verb|\dot a| \\
% \( \alpha \) \verb|\alpha| \\
% \( \epsilon \) \verb|\epsilon| \\
% \( \theta \) \verb|\theta| \\
% \( \lambda \) \verb|\lambda| \\
% \( \pi \) \verb|\pi| \\
% \( \upsilon \) \verb|\upsilon|
% \end{multicols*}

%\section{Filler material}

%\lipsum

\noindent Copyright \textcopyright{} \thedate{} \theauthor{}

\end{multicols}
\end{document}